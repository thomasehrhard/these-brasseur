% vim: set spelllang=fr:

\documentclass[a4paper,10pt]{lettre}
\usepackage[mathletters]{ucs}
\usepackage[T1]{fontenc}
\usepackage[utf8x]{inputenc}
\usepackage{graphicx}
\usepackage[french]{babel}
\usepackage{hyperref}

\begin{document}
\begin{letter}{\mbox{}}
  \address{
	\mbox{}\hskip-1cm\vskip-1em\includegraphics[width=8cm]{amu}\\
	%\mbox{}\hskip-1cm\includegraphics[width=8cm]{i2m}\\
	\smallskip\\
	Lionel Vaux Auclair\\
	Institut de Mathématiques de Marseille\\
	Site Sud, Campus de Luminy\\
	13288 MARSEILLE Cedex 9\\
        \nolinkurl{lionel.vaux@univ-amu.fr}
}
  %\anglais
  \notelephone
  \nofax
  \date{le \today}
	\signature{
	    %\vskip-4.5cm
            %\includegraphics[height=4cm]{signature-couleur}
            %\\[-.5cm]
	    Lionel Vaux Auclair\\
	    Maître de conférences (HDR) en mathématiques\\
            Université d’Aix-Marseille
	}

\makeatletter
\def\fromaddress@let@vpos{80}
\def\fromaddress@let@hpos{5}
\def\fromlieu@let@vpos{0}
\def\fromlieu@let@hpos{100}
\def\rule@vpos{-5}
\ssigwidth=8cm
\makeatother

\lieu{Marseille}
\conc{Soutien à la candidature de Léopold Brasseur pour une bourse de thèse}

\opening{Chères et chers collègues,}

\thispagestyle{empty}
\pagestyle{empty}

J’ai rencontré Léopold Brasseur à l’occasion de sa participation à la
conférence \emph{Differential \(\lambda\)-Calculus and Differential Linear
Logic, 20 Years Later} que j’ai organisée au CIRM en mai 2024,
alors qu’il était déjà en stage sous la direction de Thomas Ehrhard.

Nous avons repris contact à l’automne 2024, en vue de préciser, déjà, son
projet de thèse, portant sur les liens entre logique linéaire différentielle et
différentiation cohérente.
Ce niveau d’anticipation et de préparation (il démarrait tout juste son année
de M2) témoigne de son investissement profond dans la recherche sur ces thèmes,
à l’image des choix consistants de sujets de stages qu’il a faits jusque là.

À l’occasion de ces discussions, j’ai pu constater qu’au-delà de son intérêt
pour le sujet, Léopold Brasseur détenait déjà les savoirs et compétences clés
pour attaquer ce projet de recherche: il maîtrise les concepts et outils de la
logique linéaire qui constitue l’arrière-plan théorique du projet; et il a déjà
travaillé sur des versions fonctionnelles des deux objets d’étude
(le \(\lambda\)-calcul différentiel, et PCF différentiel cohérent).

Je soutiens donc sans réserve sa candidature et me joindrai avec enthousiasme
à l’encadrement de ses travaux.

\closing{\ }

\end{letter}
\end{document}

