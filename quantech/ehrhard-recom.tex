\documentclass[12pt,pdftex]{letter}

\usepackage{amssymb}
\usepackage[french]{babel}
\usepackage[utf8]{inputenc}
\usepackage[T1]{fontenc}
\usepackage{fullpage,irif}
%\usepackage{graphics}
\usepackage{graphicx}
\usepackage{url}

\newcommand\Corrl[1]{#1}
\newcommand\Corrt[1]{#1}

\IrifLetterhead

\begin{document}


 \address{\vspace*{2cm}\\
% Thomas Ehrhard\\
% \small Directeur de recherche au CNRS\\
% \small Directeur adjoint de l'IRIF\\
% %\small Complément
 }

%\signature{Thomas Ehrhard}
\signature{
\vspace{-2cm}
\includegraphics{s4.jpg}\\
%\vspace{0.5cm}
Thomas Ehrhard\\ 
Directeur de recherche au CNRS\\
IRIF, UMR 8243
}

\begin{letter}{}

\newcommand\Name{Léopold Brasseur}
\newcommand\Fname{Brasseur}
\newcommand\Pname{Léopold}


\opening{\textbf{Lettre de recommandation.}}

\Name{} a pris contact avec moi en 2024 pour faire un stage dans un
domaine lié à la sémantique et à la logique linéaire.
%
Je lui ai proposé de travailler sur un sujet que j'ai introduit en
2021, la différentiation cohérente, et que j'étais en train de développer
dans le cadre notamment du doctorat d'Aymeric Walch.
%
Immédiatement intéressé par ce sujet, et après une mise à niveau en
logique linéaire qui a été très rapide, \Name{} a pris ce sujet à bras
le corps et a proposé des idées très intéressantes pour comparer le
lambda-calcul différentiel et le lambda-calcul différentiel cohérent,
deux formalismes très différents alors qu'ils correspondent à des
intuitions très similaires.

À la fin de ce stage, \Name{} a décidé de suivre le M2 MPRI et de se
lancer ensuite dans une thèse sous ma direction et celle de Lionel
Vaux pour continuer et étendre cette exploration.
%
Son master se passe très bien et il a entrepris un stage à l'ENS Lyon
sous la direction de Michele Pagani et Paolo Pistone sur un sujet
étroitement lié à celui du doctorat qu'il souhaite entreprendre.

J'ai rarement eu l'occasion de rencontrer un étudiant de master aussi
profondément motivé par la recherche que \Name{}.
%
Il a également montré cet intérêt lors des deux autres stages de
recherche qu'il a entrepris, dont l'un portait sur le calcul
quantique.
%


Il me semble clair qu'il faut encourager \Name{} à
persévérer dans son projet doctorale et lui accorder un financement en
ce sens.



  \closing{\ }

\end{letter}

\end{document}



